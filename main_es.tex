%------------------------
% Latex CV Template - Ignacio Romero Aravena
%------------------------

\documentclass[letterpaper,9pt]{extarticle}

% Packages
\usepackage[utf8]{inputenc}
\usepackage{geometry}
\geometry{a4paper, margin=1in}
\usepackage{titlesec}
\usepackage{enumitem}
\usepackage{hyperref}
\usepackage{changepage}

% Formatting
\setlist{noitemsep}
\titleformat{\section}{\large\bfseries}{\thesection}{1em}{}[\titlerule]
\titlespacing*{\section}{0pt}{\baselineskip}{\baselineskip}
\newenvironment{subs}
{\adjustwidth{2em}{0pt}}
{\endadjustwidth}

%-------------------------------------------------------------------------------
\begin{document}
	
	\pagestyle{empty}
	
	% Header
	\begin{center}
		\textbf{\Large IGNACIO ROMERO ARAVENA}\\[3pt]
		\textbf{Curriculum Vitae}\\[1pt]
		\href{mailto:ignacio.romero.a@ug.uchile.cl}{ignacio.romero.a@ug.uchile.cl} | \href{https://tnacho-23.github.io}{tnacho-23.github.io} | +56 9 7131 1032
	\end{center}
	
	%------------------------
	\section*{Sobre Mí}
	Soy estudiante de Ingeniería Civil Eléctrica en la Universidad de Chile, con un fuerte interés en la robótica, la inteligencia
	computacional y el trabajo en proyectos. Me he desarrollado como líder estudiantil, ayudante de cursos universitarios y miembro
	activo en equipos de investigación y robótica. Complemento mi formación académica con competencias deportivas y apoyo a la
	comunidad. Me caracterizo por el compromiso, la proactividad y el trabajo en equipo en todos los proyectos en los que participo.
		
	%------------------------
	\section*{EDUCACIÓN Y FORMACIÓN}
	
	\noindent
	\textbf{Ingeniería Civil Eléctrica y Magíster en Cs. de la Ingeniería} \\
	2020 - Presente \\ 
	Universidad de Chile \\ 
	Facultad de Ciencias Físicas y Matemáticas \\
	\\
	
	\noindent
	\textbf{Enseñanza Media} \\
	Graduado en 2019 \\ 
	Colegio San Marcos de Arica \\ 
	Arica, Chile.
	
	%------------------------
	\section*{EXPERIENCIA}
	
	\noindent
	\textbf{Experiencia Docente} \\
	2022 - Presente \\
	Santiago, Chile \\ 
	Docente en cursos de proyecto y trabajo en equipo, inteligencia computacional y procesamiento de imágenes, y robótica y automatización. En estas instancias, me he desempeñado como profesor auxiliar y ayudante facilitando el aprendizaje en laboratorios de especialización y cursos de núcleo. Mi labor se centra en introducir metodologías de investigación, fomentar el trabajo colaborativo en proyectos de innovación y guiar el desarrollo técnico en áreas de vehículos autónomos, inteligencia artificial y programación avanzada para estudiantes de pregrado y postgrado.
	
	\noindent
	\textbf{Práctica Profesional - Advanced Mining Technology Center (AMTC)} \\
	Diciembre 2024 - Febrero 2025 \\
	Santiago, Chile \\ 
	Diseño de un sistema de clasificación de objetos utilizando aprendizaje continuo en un centro de I+D enfocado en minería. \\
	
	\noindent
	\textbf{Líder de Equipo - ElectroTutores} \\
	2024 - Presente \\
	Universidad de Chile \\ 
	Encargado de la organización docente del Departamento de Ingeniería Eléctrica y vinculación estudiantil. \\
	
	\noindent
	\textbf{Práctica Profesional - Sistemo} \\
	Enero 2023 - Febrero 2023 \\
	Santiago, Chile \\ 
	Realización de mantenciones preventivas y correctivas (software y hardware) a sistemas robóticos de logística y distribución. \\
	
	%------------------------
	\section*{PROYECTOS Y VOLUNTARIADO}
	\begin{itemize}
		\item \textbf{Uchile Robotics Team:} Participante activo en "Uchile Homebreakers" y ayudante del laboratorio de robótica.
		\item \textbf{Voluntariado Santiago 2023:} Intérprete inglés/español y gestión en la Villa Panamericana.
		\item \textbf{Inducción Universitaria FCFM:} Monitor y encargado de grupos de nuevos estudiantes (2022-2023).
		\item \textbf{Pastoral Juvenil Arica:} Coordinador de actividades voluntarias para niños y jóvenes (2015-2020).
	\end{itemize}
	
	%------------------------
	\section*{HABILIDADES Y CERTIFICACIONES}
	\begin{subs}
		\textbf{Idiomas:} Inglés Intermedio-Avanzado (TOEFL ITP: 623 pts; Cambridge FCE B2). \\
		\textbf{Habilidades:} ROS para sistemas robóticos, Programación de software, Machine Learning, Modelado e Impresión 3D, Diseño de circuitos eléctricos, Procesamiento Avanzado de imágenes. \\
		\textbf{Software:} Microsoft Office, Google Workspace, Programación de microcontroladores.
	\end{subs}
	
	%------------------------
	%\section*{REFERENCIAS}
	%\noindent
	%\textbf{Dr. Javier Ruiz del Solar} \\
	%Supervisor de Práctica y Profesor Guía de Tesis, AMTC. \\
	%\href{mailto:jruizd@ing.uchile.cl}{jruizd@ing.uchile.cl} \\
	
	%\noindent
	%\textbf{Prof. Andrés Caba Rutte} \\
	%Jefe de Estudios, Departamento de Ingeniería Eléctrica, %Universidad de Chile. \\
	%\href{mailto:acaba@ing.uchile.cl}{acaba@ing.uchile.cl}
	
\end{document}